\documentclass{article}
\usepackage{amsmath}
\usepackage{amsfonts}
\usepackage{amssymb}

\usepackage{enumitem}

\usepackage{tikz}
\usetikzlibrary{positioning}

\begin{document}

\section{}
Assume that we have some data $x_1, \dots x_n \in \mathbb{R}$. Our goal is to find a constant $b$ such that $\sum_i (y_i - b)^2$ is minimized.

\subsection{}
Find an analytic solution for the optimal value of $b$.

Consider the function
\begin{align*}
f : \mathbb{R}^n \times \mathbb{R} &\rightarrow [0, \infty) \\
(y, b) &\mapsto \sum_{i = 1}^n (y_i - b)^2.
\end{align*}
For fixed $y \in \mathbb{R}^n$ we want to solve the following minimization problem, find $b^*_y$ such that
$$
\min_{b \in \mathbb{R}} f(y, b) = f(y, b^*_y)
$$
\begin{align*}
\sum_{i = 1}^n (y_i - b)^2 = \sum_{i = 1}^n (y_i^2 - 2y_i b + b^2) = \sum_{i = 1}^n y_i^2 \underbrace{- 2 b \underline{y} + b^2}_{=: g(b)}
\end{align*}
As such we are looking for the minimum of $g(b)$. Obviously $g$ is continuously differentiable and we can calculate
$$
g'(b) = -2 \sum_{i = 1}^n y_i + 2b = 2 (b - \sum_{i = 1}^n y_i) = 0
$$
meaning that $b^* = \sum_{i = 1}^n y_i$ is the (unique!) solution to our minimization problem.

\subsection{}
How does this problem and its solution relate to the normal distribution?

\subsection{}
What if we change the loss from $\sum_i (y_i - b)^2$ to $\sum_i |y_i - b|$? Can you find the optimal solution for $b$?

Consider $b' := \overline{y}$, our new "linear" error function can be written as follows:
\begin{align*}
\sum_{i = 1}^n |y_i - b| &= \sum_{i = 1, y_i < \overline{y}}^n |y_i - \overline{y}| + \sum_{i = 1, y_i > \overline{y}}^n |y_i - \overline{y}| \\
&= \sum_{i = 1, y_i < \overline{y}}^n \overline{y} - y_i + \sum_{i = 1, y_i > \overline{y}}^n y_i - \overline{y}
\end{align*}

\section{}
Prove that the affine functions that can be expressed by $y = Wx + b$ are equivalent to linear functions on $(x, 1)$.

Consider $C := (w, 1)$ then $C^t(x, b) = (w^t|1)$ and
$$
\begin{pmatrix}
w & 1
\end{pmatrix}
\begin{pmatrix}
x \\ b
\end{pmatrix} = 
wx + b
$$

\section{}

\section{}
\subsection{}
In that case there exists $v \in \mathbb{R}^n \backslash \{0\}$ such that $X^tXv = 0$.

\subsection{}
The set of invertible matrices is dense in the set of all matrices so we'd almost certainly get an invertible matrix.

\section{}

\end{document}